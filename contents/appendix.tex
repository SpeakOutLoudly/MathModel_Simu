\chapter[附\hskip\ccwd{}\hskip\ccwd{}录]{{\heiti\zihao{3}附\hskip\ccwd{}\hskip\ccwd{}录}}

\section[\hspace{-2pt}支撑材料总览]{{\heiti\zihao{-3} \hspace{-8pt}支撑材料总览}}

本论文的所有支撑材料组织在\texttt{MathModel\_Code/}目录下,具体分类和说明如表\ref{table:supporting_materials}所示:

\begin{table}[h!]
\footnotesize
\setstretch{1.1}
\captionsetup{font={small, stretch=1.512}}
\centering
\bicaption[支撑材料分类说明]{支撑材料分类说明。}[Classification of Supporting Materials]{Classification of Supporting Materials.}
\begin{tabularx}{\textwidth}{p{1.8cm}p{4.2cm}X}
\toprule
材料类型 & 文件路径 & 说明 \\
\midrule
\multirow{3}{1.8cm}{实现代码} & \texttt{B/p1/p1.py} & 问题1:BT.2020到sRGB色域映射优化 \\
 & \texttt{B/p2/p2.py} & 问题2:四通道到五通道神经网络转换 \\
 & \texttt{B/p3/p3.py} & 问题3:LED显示器颜色校正算法 \\
\midrule
\multirow{4}{1.8cm}{原始数据} & \texttt{data/origin/xlsx/B题附件:RGB数值.xlsx} & 题目提供的原始RGB数值 \\
 & \texttt{data/preprocess/}\newline\texttt{RedPicture.xlsx} & 预处理后的红色基图数据 \\
 & \texttt{data/preprocess/}\newline\texttt{GreenPicture.xlsx} & 预处理后的绿色基图数据 \\
 & \texttt{data/preprocess/}\newline\texttt{BluePicture.xlsx} & 预处理后的蓝色基图数据 \\
\midrule
\multirow{9}{1.8cm}{结果图像} & \texttt{results/p1/DE2000.png} & 问题1:50次优化$\Delta E_{00}$分布 \\
 & \texttt{results/p1/色度.png} & 问题1:CIE1931色度图对比 \\
 & \texttt{results/p1/面积Loss.png} & 问题1:色域面积差异分析 \\
 & \texttt{results/p2/}\newline\texttt{Training\_Loss\_Curve.png} & 问题2:神经网络训练损失曲线 \\
 & \texttt{results/p2/}\newline\texttt{ΔE2000\_Error\_Histogram.png} & 问题2:色差误差分布直方图 \\
 & \texttt{results/p2/CDF.png} & 问题2:误差累积分布函数 \\
 & \texttt{results/p2/Sample.png} & 问题2:样本预测效果展示 \\
 & \texttt{results/p2/色度图.png} & 问题2:多基色色域可视化 \\
 & \texttt{results/p3/\{R,G,B\}.pdf} & 问题3:RGB三原色校正对比图 \\
\midrule
\multirow{2}{1.8cm}{环境配置} & \texttt{env.txt} & Python依赖包列表 \\
 & \texttt{mathmodel\_env.yaml} & Conda环境配置文件 \\
\midrule
说明文档 & \texttt{README.md} & 项目使用说明和运行指南 \\
\bottomrule
\end{tabularx}
\label{table:supporting_materials}
\end{table}

\lstset{
    basicstyle=\ttfamily\footnotesize,
    numbers=left,
    frame=single,
    backgroundcolor=\color{gray!10},
    showstringspaces=false,
    tabsize=2,
    captionpos=b,
    breaklines=true,                % 自动换行
    breakatwhitespace=false,        % 非空格处也能断行
    postbreak=\mbox{\textcolor{gray}{$\hookrightarrow$}\space},  % 换行标记
}
\section[\hspace{-2pt}优化函数]{{\heiti\zihao{-3} \hspace{-8pt}优化函数}}
\begin{lstlisting}[language=Python]
    import numpy as np
    import itertools
    from scipy.optimize import linprog
    import random
    from deap import base, creator, tools, algorithms
    import matplotlib.pyplot as plt
    import pandas as pd
    from typing import List, Tuple, Dict
    import warnings
    warnings.filterwarnings('ignore')
    
    # 定义适应度函数和个体(如果未定义)
    try:
        creator.create("FitnessMax", base.Fitness, weights=(1.0,))
        creator.create("Individual", list, fitness=creator.FitnessMax)
    except Exception as e:
        # 避免重复定义
        pass
    
    class TourismOptimizer:
        def __init__(self):
            # 基础数据
            self.num_spots = 6
            self.num_regions = 6
            
            # 容量数据 (按0.6折减)
            self.spot_cap = np.array([12000, 36000, 20000, 42000, 38000, 30000]) * 0.6
            self.night_cap_original = np.array([19000, 32000, 11000, 36000, 23000, 22000]) * 0.6
            self.night_cap = np.copy(self.night_cap_original) # 用于修改的容量
            self.lunch_cap = np.array([23000, 39000, 13000, 45000, 31000, 28000]) * 0.6
            
            # 区域偏好度
            self.region_prefer = np.array([7, 8, 9, 8, 6, 7])
            
            # 原始费用和时间矩阵
            self.cost_mat_raw = np.array([
                [10, 25, 30, 18, 40, 25],
                [np.inf, 10, 16, 24, 28, 18],
                [np.inf, np.inf, 10, 24, 20, 15],
                [np.inf, np.inf, np.inf, 10, 24, 16],
                [np.inf, np.inf, np.inf, np.inf, 10, 22],
                [np.inf, np.inf, np.inf, np.inf, np.inf, 10]
            ])
            
            self.time_mat_raw = np.array([
                [30, 50, 60, 35, 70, 40],
                [np.inf, 30, 25, 30, 40, 30],
                [np.inf, np.inf, 15, 35, 30, 30],
                [np.inf, np.inf, np.inf, 15, 35, 25],
                [np.inf, np.inf, np.inf, np.inf, 20, 35],
                [np.inf, np.inf, np.inf, np.inf, np.inf, 25]
            ])
            
            # 处理后的完整矩阵
            self.cost_mat = None
            self.time_mat = None
            
            # 路线相关
            self.routes_1day = []
            self.routes_2day = []
            self.routes_3day_candidates = []
            
        def preprocess_matrices(self, cost_weight=0.5, time_weight=0.5):
            """使用加权综合Floyd-Warshall算法补全费用和时间矩阵
            
            Args:
                cost_weight: 费用权重 (默认0.5)
                time_weight: 时间权重 (默认0.5)
            """
            # print(f"正在预处理交通矩阵(加权方案:费用权重={cost_weight}, 时间权重={time_weight})...") # 避免过多打印
            
            # 创建邻接矩阵 (景点-区域双向图)
            n = self.num_spots + self.num_regions  # 总节点数
            
            # 初始化矩阵
            combined_full = np.full((n, n), np.inf)
            cost_full = np.full((n, n), np.inf)
            time_full = np.full((n, n), np.inf)
            
            # 对角线为0
            np.fill_diagonal(combined_full, 0)
            np.fill_diagonal(cost_full, 0)
            np.fill_diagonal(time_full, 0)
            
            # 计算归一化常数
            finite_costs = self.cost_mat_raw[np.isfinite(self.cost_mat_raw)]
            finite_times = self.time_mat_raw[np.isfinite(self.time_mat_raw)]
            cost_max = np.max(finite_costs) if len(finite_costs) > 0 else 100
            time_max = np.max(finite_times) if len(finite_times) > 0 else 100
            
            # 填入已知的景点到区域的连接
            for r in range(self.num_regions):
                for s in range(self.num_spots):
                    if not np.isinf(self.cost_mat_raw[r, s]):
                        cost = self.cost_mat_raw[r, s]
                        time = self.time_mat_raw[r, s]
                        
                        # 归一化
                        cost_normalized = cost / cost_max
                        time_normalized = time / time_max
                        
                        # 加权综合成本
                        combined_cost = cost_weight * cost_normalized + time_weight * time_normalized
                        
                        # 景点s到区域r
                        combined_full[s, self.num_spots + r] = combined_cost
                        cost_full[s, self.num_spots + r] = cost
                        time_full[s, self.num_spots + r] = time
                        
                        # 区域r到景点s (假设双向相等)
                        combined_full[self.num_spots + r, s] = combined_cost
                        cost_full[self.num_spots + r, s] = cost
                        time_full[self.num_spots + r, s] = time
            
            # 使用综合成本运行Floyd-Warshall算法
            for k in range(n):
                for i in range(n):
                    for j in range(n):
                        if combined_full[i, k] + combined_full[k, j] < combined_full[i, j]:
                            combined_full[i, j] = combined_full[i, k] + combined_full[k, j]
                            # 同时更新对应的实际费用和时间
                            cost_full[i, j] = cost_full[i, k] + cost_full[k, j]
                            time_full[i, j] = time_full[i, k] + time_full[k, j]
            
            # 提取景点到区域的部分
            self.cost_mat = cost_full[:self.num_spots, self.num_spots:]
            self.time_mat = time_full[:self.num_spots, self.num_spots:]
            
            # print("矩阵预处理完成") # 避免过多打印
            
        def generate_1day_routes(self):
            """生成所有一日游路线: S -> R -> S"""
            # print("生成一日游路线...") # 避免过多打印
            self.routes_1day = []
            
            for s1 in range(self.num_spots):
                for r in range(self.num_regions):
                    for s2 in range(self.num_spots):
                        if s1 != s2:  # 避免重复访问同一景点
                            route = {
                                'type': '1day',
                                'path': [s1, r, s2],
                                'spots': [s1, s2],
                                'regions': [r],
                                'lunch_regions': [r],
                                'night_regions': [],
                                'cost': self.cost_mat[s1, r] + self.cost_mat[s2, r],
                                'time': self.time_mat[s1, r] + self.time_mat[s2, r],
                                'preference': sum([self.region_prefer[r]]) + 2  # 景点基础偏好
                            }
                            self.routes_1day.append(route)
            
            # print(f"一日游路线数量: {len(self.routes_1day)}") # 避免过多打印
        
        def generate_2day_routes(self, max_routes=10000):
            """生成二日游路线: S -> R -> S -> R -> S -> R -> S"""
            # print("生成二日游路线...") # 避免过多打印
            self.routes_2day = []
            count = 0
            
            for s1 in range(self.num_spots):
                for r1 in range(self.num_regions):
                    for s2 in range(self.num_spots):
                        for r2 in range(self.num_regions):
                            for s3 in range(self.num_spots):
                                for r3 in range(self.num_regions):
                                    for s4 in range(self.num_spots):
                                        if len(set([s1, s2, s3, s4])) == 4:  # 所有景点不重复
                                            if count >= max_routes:
                                                break
                                            
                                            route = {
                                                'type': '2day',
                                                'path': [s1, r1, s2, r2, s3, r3, s4],
                                                'spots': [s1, s2, s3, s4],
                                                'regions': [r1, r2, r3],
                                                'lunch_regions': [r1, r3],  # 第1天午餐r1,第2天午餐r3
                                                'night_regions': [r2],  # 第1天住宿r2
                                                'cost': (self.cost_mat[s1, r1] + self.cost_mat[s2, r1] + 
                                                         self.cost_mat[s2, r2] + self.cost_mat[s3, r2] + 
                                                         self.cost_mat[s3, r3] + self.cost_mat[s4, r3]),
                                                'time': (self.time_mat[s1, r1] + self.time_mat[s2, r1] + 
                                                         self.time_mat[s2, r2] + self.time_mat[s3, r2] + 
                                                         self.time_mat[s3, r3] + self.time_mat[s4, r3]),
                                                'preference': sum([self.region_prefer[r1], self.region_prefer[r2], 
                                                                   self.region_prefer[r3]]) + 4
                                            }
                                            self.routes_2day.append(route)
                                            count += 1
                                        if count >= max_routes:
                                            break
                                    if count >= max_routes:
                                        break
                                if count >= max_routes:
                                    break
                            if count >= max_routes:
                                break
                        if count >= max_routes:
                            break
                    if count >= max_routes:
                        break
                if count >= max_routes:
                    break
            
            # print(f"二日游路线数量: {len(self.routes_2day)}") # 避免过多打印
        
        def generate_3day_candidates_ga(self, population_size=1000, generations=50, candidate_size=5000):
            """使用遗传算法生成三日游候选路线"""
            # print("使用遗传算法生成三日游候选路线...") # 避免过多打印
            
            # 定义适应度函数 (已在类外部处理,确保只创建一次)
            
            toolbox = base.Toolbox()
            
            def create_individual():
                """创建一个个体 (三日游路线)"""
                # 随机选择6个不同的景点
                spots = random.sample(range(self.num_spots), 6)
                # 随机选择5个区域
                regions = [random.randint(0, self.num_regions-1) for _ in range(5)]
                return spots + regions
            
            def evaluate(individual):
                """评估个体的适应度"""
                spots = individual[:6]
                regions = individual[6:]
                
                # 验证景点不重复
                if len(set(spots)) != 6:
                    return (-1000,) # 惩罚重复景点路线
                
                # 计算路线: S1->R1->S2->R2->S3->R3->S4->R4->S5->R5->S6
                try:
                    cost = 0
                    time = 0
                    
                    # Day 1: S1->R1->S2->R2 (Overnight in R2)
                    cost += self.cost_mat[spots[0], regions[0]] + self.cost_mat[spots[1], regions[0]]
                    cost += self.cost_mat[spots[1], regions[1]] # Travel from S2 to R2 for overnight
                    time += self.time_mat[spots[0], regions[0]] + self.time_mat[spots[1], regions[0]]
                    time += self.time_mat[spots[1], regions[1]]
                    
                    # Day 2: S2->R2->S3->R3->S4->R4 (Overnight in R4)
                    # Travel from R2 to S3 (already at R2)
                    cost += self.cost_mat[spots[2], regions[2]]
                    cost += self.cost_mat[spots[3], regions[2]]
                    cost += self.cost_mat[spots[3], regions[3]] # Travel from S4 to R4 for overnight
                    time += self.time_mat[spots[2], regions[2]]
                    time += self.time_mat[spots[3], regions[2]]
                    time += self.time_mat[spots[3], regions[3]]
                    
                    # Day 3: S4->R4->S5->R5->S6 (End trip)
                    # Travel from R4 to S5 (already at R4)
                    cost += self.cost_mat[spots[4], regions[4]]
                    cost += self.cost_mat[spots[5], regions[4]]
                    time += self.time_mat[spots[4], regions[4]]
                    time += self.time_mat[spots[5], regions[4]]
    
                    preference = sum([self.region_prefer[r] for r in regions]) + 6 # 6 unique spots contribute to base preference
                    
                    # 归一化计算 (与MILP保持一致)
                    max_preference_3day = 9 * 5 + 6  # 理论最大偏好度
                    # 估计的最大成本和时间,用于遗传算法的归一化,不必非常精确,只要能大致区分好坏
                    max_cost_estimate = 100 * 10 # 粗略估计
                    max_time_estimate = 150 * 10 # 粗略估计
                    
                    pref_normalized = preference / max_preference_3day
                    cost_normalized = cost / max_cost_estimate
                    time_normalized = time / max_time_estimate
                    
                    # 加权综合评分 (权重与MILP一致)
                    w_pref, w_cost, w_time = 0.6, 0.2, 0.2
                    fitness = w_pref * pref_normalized - w_cost * cost_normalized - w_time * time_normalized
                    
                    return (fitness,)
                except:
                    return (-1000,) # 无效路径惩罚
            
            toolbox.register("individual", tools.initIterate, creator.Individual, create_individual)
            toolbox.register("population", tools.initRepeat, list, toolbox.individual)
            toolbox.register("evaluate", evaluate)
            
            def crossover_individual(ind1, ind2):
                """自定义交叉函数,确保景点不重复"""
                # 对景点部分使用顺序交叉(OX)
                spots1, spots2 = ind1[:6], ind2[:6]
                regions1, regions2 = ind1[6:], ind2[6:]
                
                # 顺序交叉(Order Crossover)
                def order_crossover(parent1, parent2):
                    size = len(parent1)
                    start, end = sorted(random.sample(range(size), 2))
                    
                    child = [-1] * size
                    # 复制中间段
                    child[start:end] = parent1[start:end]
                    
                    # 从parent2中按顺序填充剩余位置
                    pointer = end
                    for item in parent2[end:] + parent2[:end]:
                        if item not in child:
                            while child[pointer] != -1: # Find next empty spot
                                pointer = (pointer + 1) % size
                            child[pointer] = item
                            pointer = (pointer + 1) % size # Move pointer for next item
                    
                    return child
                
                # 对景点使用顺序交叉
                new_spots1 = order_crossover(spots1, spots2)
                new_spots2 = order_crossover(spots2, spots1)
                
                # 对区域使用简单的两点交叉
                if len(regions1) > 1:
                    cx_point = random.randint(1, len(regions1)-1)
                    new_regions1 = regions1[:cx_point] + regions2[cx_point:]
                    new_regions2 = regions2[:cx_point] + regions1[cx_point:]
                else:
                    new_regions1, new_regions2 = regions1[:], regions2[:]
                
                # 重新组合
                ind1[:] = new_spots1 + new_regions1
                ind2[:] = new_spots2 + new_regions2
                
                return ind1, ind2
            
            toolbox.register("mate", crossover_individual)
    
            def mutate_individual(individual):
                """自定义变异函数"""
                if random.random() < 0.1: # 变异概率
                    # 变异景点部分 (前6个位置) - 使用交换确保不重复
                    pos1, pos2 = random.sample(range(6), 2)
                    individual[pos1], individual[pos2] = individual[pos2], individual[pos1]
                if random.random() < 0.1: # 变异概率
                    # 变异区域部分 (后5个位置)
                    pos = random.randint(6, 10)
                    individual[pos] = random.randint(0, self.num_regions-1)
                return individual,
            
            toolbox.register("mutate", mutate_individual)
            toolbox.register("select", tools.selTournament, tournsize=3)
            
            # 运行遗传算法
            population = toolbox.population(n=population_size)
            
            # 初始评估
            fitnesses = list(map(toolbox.evaluate, population))
            for ind, fit in zip(population, fitnesses):
                ind.fitness.values = fit
            
            for gen in range(generations):
                # 选择
                offspring = toolbox.select(population, len(population))
                offspring = list(map(toolbox.clone, offspring))
                
                # 交叉和变异
                for child1, child2 in zip(offspring[::2], offspring[1::2]):
                    if random.random() < 0.5:
                        toolbox.mate(child1, child2)
                        del child1.fitness.values
                        del child2.fitness.values
                
                for mutant in offspring:
                    if random.random() < 0.2:
                        toolbox.mutate(mutant)
                        del mutant.fitness.values
                
                # 评估无效个体
                invalid_ind = [ind for ind in offspring if not ind.fitness.valid]
                fitnesses = map(toolbox.evaluate, invalid_ind)
                for ind, fit in zip(invalid_ind, fitnesses):
                    ind.fitness.values = fit
                
                population[:] = offspring # Update population with new generation
                
                # if gen % 10 == 0: # 避免过多打印
                #     fits = [ind.fitness.values[0] for ind in population if ind.fitness.valid]
                #     if fits:
                #         print(f"Generation {gen}: Max={max(fits):.2f}, Avg={np.mean(fits):.2f}")
            
            # 提取最优个体作为候选路线,保证多样性
            population.sort(key=lambda x: x.fitness.values[0], reverse=True)
            
            self.routes_3day_candidates = []
            used_route_keys = set()
            
            for ind in population:
                spots = ind[:6]
                regions = ind[6:]
                
                # 验证景点不重复
                if len(set(spots)) != 6:
                    continue  # 跳过有重复景点的个体
                
                # 创建路线唯一标识
                route_key = tuple(spots + regions)
                
                # 只选择独特的路线
                if route_key not in used_route_keys and len(self.routes_3day_candidates) < candidate_size:
                    used_route_keys.add(route_key)
                    
                    route = {
                        'type': '3day',
                        'path': [spots[0], regions[0], spots[1], regions[1], spots[2], regions[2], 
                                 spots[3], regions[3], spots[4], regions[4], spots[5]],
                        'spots': spots,
                        'regions': regions,
                        'lunch_regions': [regions[0], regions[2], regions[4]],  # 3天的午餐
                        'night_regions': [regions[1], regions[3]],  # 前2天的住宿
                        'cost': 0,  # 需要重新计算
                        'time': 0,  # 需要重新计算
                        'preference': 0,  # 需要重新计算
                        'fitness': ind.fitness.values[0]
                    }
                    
                    # 重新精确计算成本、时间、偏好
                    try:
                        cost = (self.cost_mat[spots[0], regions[0]] + self.cost_mat[spots[1], regions[0]] +
                               self.cost_mat[spots[1], regions[1]] + self.cost_mat[spots[2], regions[1]] +
                               self.cost_mat[spots[2], regions[2]] + self.cost_mat[spots[3], regions[2]] +
                               self.cost_mat[spots[3], regions[3]] + self.cost_mat[spots[4], regions[3]] +
                               self.cost_mat[spots[4], regions[4]] + self.cost_mat[spots[5], regions[4]])
                        
                        time = (self.time_mat[spots[0], regions[0]] + self.time_mat[spots[1], regions[0]] +
                               self.time_mat[spots[1], regions[1]] + self.time_mat[spots[2], regions[1]] +
                               self.time_mat[spots[2], regions[2]] + self.time_mat[spots[3], regions[2]] +
                               self.time_mat[spots[3], regions[3]] + self.time_mat[spots[4], regions[3]] +
                               self.time_mat[spots[4], regions[4]] + self.time_mat[spots[5], regions[4]])
                        
                        preference = sum([self.region_prefer[r] for r in regions]) + 6
                        
                        route['cost'] = cost
                        route['time'] = time
                        route['preference'] = preference
                        
                        self.routes_3day_candidates.append(route)
                    except:
                        continue
            
            # print(f"三日游候选路线数量: {len(self.routes_3day_candidates)}") # 避免过多打印
        
        def generate_3day_routes_hybrid(self, target_routes=5000):
            """混合方法生成三日游路线:遗传算法 + 启发式规则"""
            print("使用混合方法生成三日游路线...")
            
            self.routes_3day_candidates = []
            
            # 1. 先用遗传算法生成高质量路线
            print("步骤1: 遗传算法生成高质量路线...")
            self.generate_3day_candidates_ga(candidate_size=int(target_routes * 0.1), population_size=1000, generations=50) # 减少GA生成的比例
            ga_routes_count = len(self.routes_3day_candidates)
            print(f"遗传算法生成: {ga_routes_count}条路线")
            
            # 2. 启发式生成多样化路线
            print("步骤2: 启发式生成多样化路线...")
            used_route_keys = set()
            for route in self.routes_3day_candidates:
                route_key = tuple(route['spots'] + route['regions'])
                used_route_keys.add(route_key)
            
            # 为每个区域组合生成路线
            # 考虑更系统地生成,而不是固定组合
            # 随机生成区域组合,增加多样性
            
            num_heuristic_routes = target_routes - ga_routes_count
            
            # 尝试生成更多样化的区域组合
            for _ in range(num_heuristic_routes * 2): # 尝试生成更多,因为会有重复或无效
                if len(self.routes_3day_candidates) >= target_routes:
                    break
                
                # 随机选择5个区域,可以重复
                region_combo = [random.randint(0, self.num_regions - 1) for _ in range(5)]
                
                # 为该区域组合生成多种景点排列 (随机选择部分)
                spot_permutations = list(itertools.permutations(range(self.num_spots)))
                selected_perms = random.sample(spot_permutations, min(2, len(spot_permutations))) # 减少每个区域组合的景点排列数
                
                for spots in selected_perms:
                    if len(self.routes_3day_candidates) >= target_routes:
                        break
                        
                    route_key = tuple(list(spots) + region_combo)
                    if route_key not in used_route_keys:
                        used_route_keys.add(route_key)
                        
                        route = {
                            'type': '3day',
                            'path': [spots[0], region_combo[0], spots[1], region_combo[1], spots[2], region_combo[2], 
                                     spots[3], region_combo[3], spots[4], region_combo[4], spots[5]],
                            'spots': list(spots),
                            'regions': region_combo,
                            'lunch_regions': [region_combo[0], region_combo[2], region_combo[4]],
                            'night_regions': [region_combo[1], region_combo[3]],
                        }
                        
                        # 计算成本、时间、偏好
                        try:
                            cost = (self.cost_mat[spots[0], region_combo[0]] + self.cost_mat[spots[1], region_combo[0]] +
                                   self.cost_mat[spots[1], region_combo[1]] + self.cost_mat[spots[2], region_combo[1]] +
                                   self.cost_mat[spots[2], region_combo[2]] + self.cost_mat[spots[3], region_combo[2]] +
                                   self.cost_mat[spots[3], region_combo[3]] + self.cost_mat[spots[4], region_combo[3]] +
                                   self.cost_mat[spots[4], region_combo[4]] + self.cost_mat[spots[5], region_combo[4]])
                            
                            time = (self.time_mat[spots[0], region_combo[0]] + self.time_mat[spots[1], region_combo[0]] +
                                   self.time_mat[spots[1], region_combo[1]] + self.time_mat[spots[2], region_combo[1]] +
                                   self.time_mat[spots[2], region_combo[2]] + self.time_mat[spots[3], region_combo[2]] +
                                   self.time_mat[spots[3], region_combo[3]] + self.time_mat[spots[4], region_combo[3]] +
                                   self.time_mat[spots[4], region_combo[4]] + self.time_mat[spots[5], region_combo[4]])
                            
                            preference = sum([self.region_prefer[r] for r in region_combo]) + 6
                            
                            route['cost'] = cost
                            route['time'] = time
                            route['preference'] = preference
                            
                            self.routes_3day_candidates.append(route)
                        except:
                            continue
            
            print(f"混合方法总共生成: {len(self.routes_3day_candidates)}条路线")
            print(f"其中遗传算法: {ga_routes_count}条,启发式: {len(self.routes_3day_candidates) - ga_routes_count}条")
        
        def solve_milp(self, routes, total_tourists=10000, weights=(0.6, 0.2, 0.2)):
            """使用混合整数线性规划求解路线分配"""
            # print(f"求解MILP,路线数量: {len(routes)}") # 避免过多打印
            
            if not routes:
                return {'solution': [], 'total_satisfaction': 0, 'total_tourists': 0} # 返回包含总满意度和总游客数的字典
            
            n_routes = len(routes)
            w_pref, w_cost, w_time = weights
            
            # 计算归一化的理论最大值
            max_preference_1day = 9 * 1 + 2  # 1个区域 + 2个景点
            max_preference_2day = 9 * 3 + 4  # 3个区域 + 4个景点  
            max_preference_3day = 9 * 5 + 6  # 5个区域 + 6个景点
            
            # 成本和时间:从所有路线中找最大值作为归一化基准
            max_cost = max([route['cost'] for route in routes]) if routes else 1
            max_time = max([route['time'] for route in routes]) if routes else 1
            
            # 根据路线类型确定偏好度最大值
            def get_max_preference(route_type):
                if route_type == '1day':
                    return max_preference_1day
                elif route_type == '2day':
                    return max_preference_2day
                else:  # 3day
                    return max_preference_3day
            
            # 构建目标函数系数 (最大化满意度 = 最大化偏好 - 最小化成本和时间)
            c = []
            for route in routes:
                # 归一化各项指标到[0,1]范围
                pref_normalized = route['preference'] / get_max_preference(route['type'])
                cost_normalized = route['cost'] / max_cost
                time_normalized = route['time'] / max_time
                
                # 加权计算满意度
                satisfaction = w_pref * pref_normalized - w_cost * cost_normalized - w_time * time_normalized
                c.append(-satisfaction)  # linprog求最小值,所以取负
            
            # 约束矩阵
            A_ub = []
            b_ub = []
            
            # 景点容量约束 - 按时段分别约束
            max_days = 3  # 最多3日游
            max_time_slots = max_days * 2  # 每天上午+下午 (粗略估计,实际应更精细,但为了简化模型,保持原逻辑)
            
            for s in range(self.num_spots):
                constraint = []
                for route in routes:
                    # 计算该路线中景点s被访问的次数
                    visit_count = route['spots'].count(s)
                    constraint.append(visit_count)
                A_ub.append(constraint)
                # 景点在多个时段可复用,总容量 = 单时段容量 × 时段数
                total_capacity = self.spot_cap[s] * max_time_slots
                b_ub.append(total_capacity)
            
            # 午餐容量约束
            for r in range(self.num_regions):
                constraint = []
                for route in routes:
                    count = route['lunch_regions'].count(r)
                    constraint.append(count)
                A_ub.append(constraint)
                b_ub.append(self.lunch_cap[r])
            
            # 住宿容量约束 - 按夜晚时段复用
            max_nights = max_days - 1  # 最大夜晚数(3日游需要2晚)
            
            for r in range(self.num_regions):
                constraint = []
                for route in routes:
                    count = route['night_regions'].count(r)
                    constraint.append(count)
                A_ub.append(constraint)
                # 住宿容量可以在不同夜晚复用
                total_night_capacity = self.night_cap[r] * max_nights # 使用 self.night_cap
                b_ub.append(total_night_capacity)
            
            # 等式约束:总游客数
            A_eq = [[1] * n_routes]
            b_eq = [total_tourists]
            
            # 变量边界
            bounds = [(0, total_tourists) for _ in range(n_routes)]
            
            try:
                # 求解
                result = linprog(c, A_ub=A_ub, b_ub=b_ub, A_eq=A_eq, b_eq=b_eq, 
                                 bounds=bounds, method='highs')
                
                if result.success:
                    solution = []
                    total_s = 0
                    total_t = 0
                    for i, x in enumerate(result.x):
                        if x > 0.1:  # 过滤掉很小的值
                            num_tourists = int(round(x))
                            route_info = routes[i].copy()
                            route_info['tourists'] = num_tourists
                            # 重新计算归一化满意度用于显示
                            pref_normalized = route_info['preference'] / get_max_preference(route_info['type'])
                            cost_normalized = route_info['cost'] / max_cost
                            time_normalized = route_info['time'] / max_time
                            route_info['satisfaction'] = w_pref * pref_normalized - w_cost * cost_normalized - w_time * time_normalized
                            solution.append(route_info)
                            total_s += num_tourists * route_info['satisfaction']
                            total_t += num_tourists
                    
                    return {'solution': solution, 'total_satisfaction': total_s, 'total_tourists': total_t}
                else:
                    # print("MILP求解失败") # 避免过多打印
                    return {'solution': [], 'total_satisfaction': 0, 'total_tourists': 0}
            except Exception as e:
                # print(f"MILP求解出错: {e}") # 避免过多打印
                return {'solution': [], 'total_satisfaction': 0, 'total_tourists': 0}
        
        def run_optimization(self, cost_weight=0.5, time_weight=0.5, tourists_1day=30000, tourists_2day=30000, tourists_3day=20000):
            """运行完整的优化流程
            
            Args:
                cost_weight: 费用权重,用于路径预处理 (默认0.5)
                time_weight: 时间权重,用于路径预处理 (默认0.5)
                tourists_1day: 一日游总游客数
                tourists_2day: 二日游总游客数
                tourists_3day: 三日游总游客数
            """
            # print("开始旅游路线优化...") # 避免过多打印
            # print(f"使用权重配置:费用权重={cost_weight}, 时间权重={time_weight}")
            
            # 1. 数据预处理
            self.preprocess_matrices(cost_weight, time_weight)
            
            # 2. 生成路线
            self.generate_1day_routes()
            self.generate_2day_routes(max_routes=5000)  # 限制二日游路线数量
            self.generate_3day_routes_hybrid(target_routes=5000) # 使用混合方法生成三日游路线
            
            # 3. 分别求解三种旅游方案
            results = {}
            
            # 一日游
            # print("\n=== 求解一日游方案 ===")
            solution_1day_res = self.solve_milp(self.routes_1day, total_tourists=tourists_1day)
            results['1day'] = solution_1day_res['solution']
            results['1day_metrics'] = {'total_satisfaction': solution_1day_res['total_satisfaction'], 'total_tourists': solution_1day_res['total_tourists']}
            
            # 二日游
            # print("\n=== 求解二日游方案 ===")
            solution_2day_res = self.solve_milp(self.routes_2day, total_tourists=tourists_2day)
            results['2day'] = solution_2day_res['solution']
            results['2day_metrics'] = {'total_satisfaction': solution_2day_res['total_satisfaction'], 'total_tourists': solution_2day_res['total_tourists']}
            
            # 三日游
            # print("\n=== 求解三日游方案 ===")
            solution_3day_res = self.solve_milp(self.routes_3day_candidates, total_tourists=tourists_3day)
            results['3day'] = solution_3day_res['solution']
            results['3day_metrics'] = {'total_satisfaction': solution_3day_res['total_satisfaction'], 'total_tourists': solution_3day_res['total_tourists']}
            
            return results
        
        def get_aggregated_metrics(self, results):
            """从优化结果中聚合总满意度和总游客数"""
            total_satisfaction_all_types = 0
            total_tourists_all_types = 0
    
            for day_type in ['1day', '2day', '3day']:
                if f'{day_type}_metrics' in results:
                    total_satisfaction_all_types += results[f'{day_type}_metrics']['total_satisfaction']
                    total_tourists_all_types += results[f'{day_type}_metrics']['total_tourists']
            
            return total_satisfaction_all_types, total_tourists_all_types
    
        def print_results(self, results):
            """打印优化结果"""
            for day_type, solution in results.items():
                if not isinstance(solution, list): # 过滤掉 metrics 字典
                    continue
    
                print(f"\n{'='*50}")
                print(f"{day_type.upper()} 旅游方案结果")
                print(f"{'='*50}")
                
                if not solution:
                    print("无可行解")
                    continue
                
                total_tourists = sum([route['tourists'] for route in solution])
                total_cost = sum([route['tourists'] * route['cost'] for route in solution])
                total_time = sum([route['tourists'] * route['time'] for route in solution])
                total_satisfaction = sum([route['tourists'] * route['satisfaction'] for route in solution])
                
                print(f"路线类型数量: {len(solution)}")
                print(f"总游客数: {total_tourists}")
                print(f"总交通成本: {total_cost:.0f} 元")
                print(f"总交通时间: {total_time:.0f} 分钟")
                print(f"总满意度: {total_satisfaction:.2f}")
                print(f"平均满意度: {total_satisfaction/total_tourists:.2f}")
                
                print("\n具体路线分配:")
                for i, route in enumerate(solution[:10]):  # 只显示前10个
                    spots_str = "->".join([f"S{s+1}" for s in route['spots']])
                    regions_str = "->".join([f"R{r+1}" for r in route['regions']])
                    print(f"路线{i+1}: {route['tourists']}人")
                    print(f"   景点路径: {spots_str}")
                    print(f"   区域路径: {regions_str}")
                    print(f"   成本: {route['cost']:.1f}元, 时间: {route['time']:.1f}分钟, 偏好: {route['preference']}")
                    print()
    
    # 扩容成本函数
    def expansion_cost(delta_K: float) -> float:
        """
        扩容成本函数
        Args:
            delta_K: 新增接待能力(万人次)
        Returns:
            扩容成本(亿元)
        """
        c1 = 0.0007  # 亿元
        gamma = 1.09
        return c1 * (delta_K ** gamma)
    
    # 净收益评价函数 (在主脚本中调用)
    # B(delta_K) = vs * [Z(K) - Z(K0)] + vp * [N(K) - N(K0)]
    # F(delta_K) = B(delta_K) - C(delta_K)

\end{lstlisting}

\section[\hspace{-2pt}运行函数]{{\heiti\zihao{-3} \hspace{-8pt}运行函数}}

\begin{lstlisting}[language=Python]
    #!/usr/bin/env python3
    # -*- coding: utf-8 -*-
    """
    旅游路线优化运行脚本
    """
    import deap
    from tourism_optimization import TourismOptimizer, expansion_cost # Import expansion cost function
    import numpy as np # Import numpy
    import matplotlib.pyplot as plt
    import matplotlib # Import matplotlib to configure fonts
    
    # --- Start: Matplotlib Chinese Font Configuration ---
    # Configure font to support Chinese characters
    # Try 'SimHei' first, if not available, try 'Microsoft YaHei' or 'WenQuanYi Micro Hei'
    matplotlib.rcParams['font.sans-serif'] = ['SimHei', 'Microsoft YaHei', 'Arial Unicode MS'] 
    matplotlib.rcParams['axes.unicode_minus'] = False # Solve the problem of '-' displaying as a square
    # --- End: Matplotlib Chinese Font Configuration ---
    
    def main():
        print("=" * 60)
        print("重庆旅游路线优化系统")
        print("=" * 60)
        
        # Create optimizer instance
        optimizer = TourismOptimizer()
        
        # Run optimization
        try:
            # Test different weight configurations
            weight_configs = [
                (0.5, 0.5, "均衡型"),
                # (0.7, 0.3, "费用优先型"), # Can comment out other configurations for faster expansion analysis
                # (0.3, 0.7, "时间优先型")
            ]
            
            all_results = {}
            
            for cost_w, time_w, config_name in weight_configs:
                print(f"\n{'='*80}")
                print(f"运行{config_name}配置 (费用权重={cost_w}, 时间权重={time_w})")
                print(f"{'='*80}")
                
                # Run optimization and get results
                results = optimizer.run_optimization(cost_weight=cost_w, time_weight=time_w)
                all_results[config_name] = results
                
                # Print results
                optimizer.print_results(results)
                
                # Save results to file
                save_results_to_file(results, suffix=f"_{config_name}")
            
            # Compare results of different configurations
            compare_configurations(all_results)
    
            # ====== Problem 3: New Hotel Expansion Analysis ======
            print(f"\n{'='*80}")
            print("问题3:新建旅馆扩容分析")
            print(f"{'='*80}")
            analyze_hotel_expansion(optimizer)
            
        except Exception as e:
            print(f"优化过程中出现错误: {e}")
            import traceback
            traceback.print_exc()
    
    def compare_configurations(all_results):
        """比较不同权重配置的结果"""
        print(f"\n{'='*80}")
        print("不同权重配置结果比较")
        print(f"{'='*80}")
        
        for day_type in ['1day', '2day', '3day']:
            print(f"\n{day_type.upper()} 方案比较:")
            print("-" * 60)
            print(f"{'配置':<12} {'路线数':<8} {'总游客':<8} {'总费用':<12} {'总时间':<12} {'平均满意度':<12}")
            print("-" * 60)
            
            for config_name, results in all_results.items():
                solution = results.get(day_type, [])
                if solution:
                    total_tourists = sum([route['tourists'] for route in solution])
                    total_cost = sum([route['tourists'] * route['cost'] for route in solution])
                    total_time = sum([route['tourists'] * route['time'] for route in solution])
                    total_satisfaction = sum([route['tourists'] * route['satisfaction'] for route in solution])
                    avg_satisfaction = total_satisfaction / total_tourists if total_tourists > 0 else 0
                    
                    print(f"{config_name:<12} {len(solution):<8} {total_tourists:<8} {total_cost:<12.0f} "
                          f"{total_time:<12.0f} {avg_satisfaction:<12.2f}")
                else:
                    print(f"{config_name:<12} {'无解':<8} {'-':<8} {'-':<12} {'-':<12} {'-':<12}")
    
    
    def save_results_to_file(results, suffix=""):
        """将结果保存到文件"""
        filename = f'optimization_results{suffix}.txt'
        with open(filename, 'w', encoding='utf-8') as f:
            f.write("重庆旅游路线优化结果\n")
            f.write("=" * 50 + "\n\n")
            
            for day_type_key, solution_or_metrics in results.items():
                if not isinstance(solution_or_metrics, list): # Filter out metrics dictionary
                    continue
                solution = solution_or_metrics
    
                day_type = day_type_key.replace('_metrics', '') # Get actual day_type
                
                f.write(f"{day_type.upper()} 旅游方案结果\n")
                f.write("=" * 30 + "\n")
                
                if not solution:
                    f.write("无可行解\n\n")
                    continue
                
                total_tourists = sum([route['tourists'] for route in solution])
                total_cost = sum([route['tourists'] * route['cost'] for route in solution])
                total_time = sum([route['tourists'] * route['time'] for route in solution])
                total_satisfaction = sum([route['tourists'] * route['satisfaction'] for route in solution])
                
                f.write(f"路线类型数量: {len(solution)}\n")
                f.write(f"总游客数: {total_tourists}\n")
                f.write(f"总交通成本: {total_cost:.0f} 元\n")
                f.write(f"总交通时间: {total_time:.0f} 分钟\n")
                f.write(f"总满意度: {total_satisfaction:.2f}\n")
                f.write(f"平均满意度: {total_satisfaction/total_tourists:.2f}\n\n")
                
                f.write("具体路线分配:\n")
                for i, route in enumerate(solution[:10]):
                    spots_str = "->".join([f"S{s+1}" for s in route['spots']])
                    regions_str = "->".join([f"R{r+1}" for r in route['regions']])
                    f.write(f"路线{i+1}: {route['tourists']}人\n")
                    f.write(f"  景点路径: {spots_str}\n")
                    f.write(f"  区域路径: {regions_str}\n")
                    f.write(f"  成本: {route['cost']:.1f}元, 时间: {route['time']:.1f}分钟, 偏好: {route['preference']}\n\n")
                f.write("\n")
        
        print(f"\n结果已保存到 {filename} 文件")
    
    
    def analyze_hotel_expansion(optimizer: TourismOptimizer):
        """
        Analyzes the hotel expansion problem to find the optimal expansion area and scale.
        """
        print("\n开始分析旅馆扩容方案...")
    
        # Define economic value parameters
        # vs: Economic value per unit satisfaction (billion CNY/satisfaction)
        # vp: Average profit per tourist (billion CNY/person)
        v_s_per_million_satisfaction = 0.1 # Assume 1 million satisfaction brings 0.1 billion CNY
        v_p_per_million_tourists = 0.01 # Assume 1 million tourists brings 0.01 billion CNY (10 CNY/person, 1 million * 10 = 10 million CNY = 0.01 billion CNY)
    
        # Run baseline optimization to get metrics at K0
        print("获取基准指标 (扩容前)...")
        base_results = optimizer.run_optimization(cost_weight=0.5, time_weight=0.5, tourists_1day=30000, tourists_2day=30000, tourists_3day=20000)
        Z_K0_total, N_K0_total = optimizer.get_aggregated_metrics(base_results)
        
        # Convert number of tourists to millions to match delta_K unit
        N_K0_total_wan = N_K0_total / 10000.0
    
        print(f"基准总满意度 Z(K0): {Z_K0_total:.2f}")
        print(f"基准总游客数 N(K0): {N_K0_total_wan:.2f} 万人次")
    
        best_net_benefit = -np.inf
        best_region_idx = -1
        best_delta_K_wan = 0
        
        # Define expansion capacity range (in millions of persons)
        # Consider from 0 to a reasonable maximum value, e.g., twice the original maximum capacity
        max_delta_K_wan = max(optimizer.night_cap_original) / 10000.0 * 2 # Twice the original max capacity, in millions of persons
        delta_K_wan_range = np.arange(0, max_delta_K_wan + 1, 0.5) # From 0 to max_delta_K_wan, step 0.5 million persons
        
        # Store all test results
        expansion_results = []
    
        print("\n迭代各区域及扩容规模进行模拟...")
        for r_idx in range(optimizer.num_regions):
            original_night_cap_r = optimizer.night_cap_original[r_idx] # Get the original capacity of this region
    
            for delta_K_wan in delta_K_wan_range:
                current_night_cap_r = original_night_cap_r + (delta_K_wan * 10000) # Convert millions of persons back to persons
                
                # Temporarily modify the accommodation capacity of this region
                optimizer.night_cap[r_idx] = current_night_cap_r
                
                # Rerun optimization
                # Note: run_optimization will regenerate routes, ensuring each run is based on the modified capacity
                current_results = optimizer.run_optimization(
                    cost_weight=0.5, time_weight=0.5, # Use balanced weights for expansion analysis
                    tourists_1day=30000, tourists_2day=30000, tourists_3day=20000
                )
                Z_K_total, N_K_total = optimizer.get_aggregated_metrics(current_results)
                N_K_total_wan = N_K_total / 10000.0
    
                # Calculate benefit B(delta_K)
                B_delta_K = (v_s_per_million_satisfaction * (Z_K_total - Z_K0_total) + 
                             v_p_per_million_tourists * (N_K_total_wan - N_K0_total_wan))
                
                # Calculate cost C(delta_K)
                C_delta_K = expansion_cost(delta_K_wan)
                
                # Calculate net benefit F(delta_K)
                F_delta_K = B_delta_K - C_delta_K
                
                expansion_results.append({
                    'region_idx': r_idx,
                    'region_name': f'区域{r_idx + 1}',
                    'delta_K_wan': delta_K_wan,
                    'original_night_cap_wan': original_night_cap_r / 10000.0,
                    'new_night_cap_wan': current_night_cap_r / 10000.0,
                    'Z_K_total': Z_K_total,
                    'N_K_total_wan': N_K_total_wan,
                    'Benefit_B': B_delta_K,
                    'Cost_C': C_delta_K,
                    'Net_Benefit_F': F_delta_K
                })
    
                # Update optimal solution
                if F_delta_K > best_net_benefit:
                    best_net_benefit = F_delta_K
                    best_region_idx = r_idx
                    best_delta_K_wan = delta_K_wan
    
            # Restore the original capacity of this region for the next region's test
            optimizer.night_cap[r_idx] = optimizer.night_cap_original[r_idx]
    
        print("\n扩容分析结果:")
        print("-" * 60)
        print(f"{'区域':<8} {'扩容规模(万人次)':<18} {'原容量(万人次)':<16} {'新容量(万人次)':<16} {'净收益(亿元)':<12}")
        print("-" * 60)
        
        # Print the top 20 net benefit solutions
        sorted_expansion_results = sorted(expansion_results, key=lambda x: x['Net_Benefit_F'], reverse=True)
        for i, res in enumerate(sorted_expansion_results[:20]):
            print(f"{res['region_name']:<8} {res['delta_K_wan']:<18.2f} {res['original_night_cap_wan']:<16.2f} {res['new_night_cap_wan']:<16.2f} {res['Net_Benefit_F']:<12.4f}")
    
        print(f"\n最终建议:")
        if best_region_idx != -1:
            print(f"建议扩容区域: 区域{best_region_idx + 1}")
            print(f"建议扩容规模: {best_delta_K_wan:.2f} 万人次")
            print(f"最大净收益: {best_net_benefit:.4f} 亿元")
            print(f"该区域原有住宿接待能力: {optimizer.night_cap_original[best_region_idx]/10000:.2f} 万人次")
            print(f"该区域扩容后住宿接待能力: {(optimizer.night_cap_original[best_region_idx] + best_delta_K_wan * 10000)/10000:.2f} 万人次")
    
            # Plot net benefit
            plt.figure(figsize=(12, 6))
            
            region_colors = plt.cm.tab10 # Use matplotlib's color map
            
            # Filter data for each region
            regions_data = {r_idx: [] for r_idx in range(optimizer.num_regions)}
            for res in expansion_results:
                regions_data[res['region_idx']].append(res)
            
            for r_idx, data in regions_data.items():
                data.sort(key=lambda x: x['delta_K_wan'])
                delta_K_wan_values = [d['delta_K_wan'] for d in data]
                net_benefit_values = [d['Net_Benefit_F'] for d in data]
                
                plt.plot(delta_K_wan_values, net_benefit_values, marker='o', linestyle='-',
                         label=f'区域 {r_idx + 1}', color=region_colors(r_idx))
            
            plt.title('不同区域及扩容规模下的净收益')
            plt.xlabel('新增接待能力 (万人次)')
            plt.ylabel('净收益 (亿元)')
            plt.grid(True)
            plt.legend(title='区域')
            plt.axhline(0, color='grey', linestyle='--', linewidth=0.8) # Zero benefit line
            plt.scatter(best_delta_K_wan, best_net_benefit, color='red', marker='X', s=200, 
                        label=f'最优解: 区域{best_region_idx+1}, {best_delta_K_wan:.2f}万人次, {best_net_benefit:.4f}亿元')
            plt.legend()
            plt.tight_layout()
            plt.savefig('net_benefit_vs_expansion.png')
            plt.show()
    
        else:
            print("未找到可行的扩容方案,或者所有扩容方案都导致负净收益。")
    
    
    if __name__ == "__main__":
        main()
    
\end{lstlisting}

