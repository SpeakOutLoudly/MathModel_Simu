\chapter[\hspace{0pt}问题分析]{{\heiti\zihao{3}\hspace{0pt}问题分析}}\label{chapter1: 问题分析}
\removelofgap
\removelotgap

\section[\hspace{-2pt}问题一]{{\heiti\zihao{-3} \hspace{-8pt}问题一}}\label{section1: 问题一}

针对问题一,我们首先需要将景点吸引力通过合适的数学函数进行量化,该函数需要综合景点可达区域、区域偏好、可容纳人数等影响。以此作为优化目标函数的一个因子。此外我们还需要考虑到不同景点、区域选择会带来不同的时间、交通成本,应当为其增加约束条件并作为优化函数的另一个因子。由于不同旅游天数会对问题产生较大的影响。一日游在晚上不必前往旅馆,而二日游、三日游都需要前往旅馆。因此我们需要额外考虑不同天数下的交通问题。

对于表3中空白的区域,我们假设可以通过中转的方式到达。比如,从景点TODO到区域TODO,以其为中转,再到景区TODO。我们应当采用Floyd‑Warshall算法计算不可直达的两景区之间的最短距离。在此基础上,采用混合整数线性优化方法得到最优方案。

我们提出一种结合景点吸引力、游客偏好、时间与空间约束的旅游路线规划模型。

\section[\hspace{-2pt}问题二]{{\heiti\zihao{-3} \hspace{-8pt}问题二}}\label{section1: 问题二}

在问题二中,当旅游套餐总数被限定在 10 种以内时,原先追求满意度极大化的模型必须在游客体验与管理简化之间做出权衡。为此1,我们提出了基于旅游路线的K均值算法。该算法采用多种因素加权的距离函数判断两个旅游路线之间的相似度。该函数综合考虑到方案间的景点、路线、区域等相似度,将重复性较强的方案合并。

通过这种方法,我们可以得到5-6个聚类中心,而每个聚类中心包含的方案数量在2-5个之间。再根据满意度排序,舍弃满意度较低的几个方案,最终将总数控制到10以内。通过K均值算法,我们可以很好地保证方案总体满意度较高的同时,满足多样化需求,满足方案总数的限制。

\section[\hspace{-2pt}问题三]{{\heiti\zihao{-3} \hspace{-8pt}问题三}}\label{section1: 问题三}



