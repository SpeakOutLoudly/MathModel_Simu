\chapter[\hspace{0pt}问题分析]{{\heiti\zihao{3}\hspace{0pt}问题分析}}\label{chapter1: 问题分析}
\removelofgap
\removelotgap

\section[\hspace{-2pt}问题一]{{\heiti\zihao{-3} \hspace{-8pt}问题一}}\label{section1: 问题一}

针对问题一,我们首先需要将景点吸引力通过合适的数学函数进行量化,该函数需要综合景点可达区域、区域偏好、可容纳人数等影响。以此作为优化目标函数的一个因子。此外我们还需要考虑到不同景点、区域选择会带来不同的时间、交通成本,应当为其增加约束条件并作为优化函数的另一个因子。由于不同旅游天数会对问题产生较大的影响。一日游在晚上不必前往旅馆,而二日游、三日游都需要前往旅馆。因此我们需要额外考虑不同天数下的交通问题。

对于表3中空白的区域,我们假设可以通过中转的方式到达。对此,我们应当采用Floyd‑Warshall算法计算不可直达的两景区之间的最短距离。在此基础上,采用混合整数线性优化方法得到最优方案。

我们提出一种结合景点吸引力、游客偏好、时间与空间约束的旅游路线规划模型。

\section[\hspace{-2pt}问题二]{{\heiti\zihao{-3} \hspace{-8pt}问题二}}\label{section1: 问题二}

在问题二中,当旅游套餐总数被限定在 10 种以内时,原先追求满意度极大化的模型必须在游客体验与管理简化之间做出权衡。为此1,我们提出了基于旅游路线的K均值算法。该算法采用多种因素加权的距离函数判断两个旅游路线之间的相似度。该函数综合考虑到方案间的景点、路线、区域等相似度,将重复性较强的方案合并。

通过这种方法,我们可以得到5-6个聚类中心,而每个聚类中心包含的方案数量在2-5个之间。再根据满意度排序,舍弃满意度较低的几个方案,最终将总数控制到10以内。通过K均值算法,我们可以很好地保证方案总体满意度较高的同时,满足多样化需求,满足方案总数的限制。

\section[\hspace{-2pt}问题三]{{\heiti\zihao{-3} \hspace{-8pt}问题三}}\label{section1: 问题三}

在问题三中,我们的目标是为旅游集团确定最优的\textbf{住宿扩容区域以及扩容规模},以在满足游客需求的同时最大化经济效益。与问题一、二侧重于既有容量条件下的线路设计不同,问题三需要在\textbf{提升接待能力}与\textbf{追加建设成本}之间进行权衡。

首先,根据\textbf{模型假设}(见 \ref{chapter2:模型假设}),景区夜宿容量可以通过追加投资进行扩容,且扩容成本呈现规模报酬递减特征。结合题目背景,我们应当引入如下两类关键因素:扩容收益与扩容成本。扩容收益由扩容后游客满意度增量与额外可接待的游客数量增量组成。扩容成本采用经验函数 $C(\Delta K)=c_{1}(\Delta K)^{\gamma}$,体现越大规模扩建边际成本递增的现实规律。

% 1. \textbf{扩容收益 $B(\Delta K)$}——由两部分组成:① 扩容后游客满意度增量 $Z(K)-Z(K^{0})$,反映由于容量瓶颈缓解而能分配到更优路线的游客占比上升;② 额外可接待的游客数量增量 $N(K)-N(K^{0})$。二者分别以权重 $v_s,\,v_p$ 衡量其对集团收入的影响。

% 2. \textbf{扩容成本 $C(\Delta K)$}——采用经验函数 $C(\Delta K)=c_{1}(\Delta K)^{\gamma}$,其中 $c_{1}=7\times10^{-4}$ 亿元,$\gamma=1.09$,体现越大规模扩建边际成本递增的现实规律。

% 据此可得净收益函数
% \begin{equation}
%   F_r(\Delta K)=v_s\bigl[Z_r(K)-Z_r(K^{0})\bigr]+v_p\bigl[N_r(K)-N_r(K^{0})\bigr]-C(\Delta K),
%\end{equation}
%其中下标 $r$ 表示对第 $r$ 个区域进行扩容。

为了求解最优 $(r^{*},\Delta K^{*})$,我们可以采用\textbf{双层优化策略}:

\begin{itemize}
  \item \textbf{外层枚举}\;六个候选区域及一系列离散的扩容步长 $\Delta\mathcal K$(如 0, 0.5, 1, \dots)。
  \item \textbf{内层优化}\;在给定 $(r,\Delta K)$ 时,调用问题一中“枚举路线 + MILP”框架(辅以遗传算法与启发式多样化规则)重新分配游客,得到 $Z_r(K),\,N_r(K)$。
\end{itemize}

将外层搜索与内层分配结合,即可计算所有候选方案的净收益 $F_r(\Delta K)$,从而选出
\[
  (r^{*},\Delta K^{*})=\arg\max_{r,\,\Delta K} F_r(\Delta K).
\]

最后,绘制 $F_r(\Delta K)$ 随扩容规模变化的曲线,并给出最优扩容区域及规模。这不仅为政府和企业提供了量化的建设决策依据,也直观展示了不同投入水平下的成本–收益权衡。



