\chapter[\hspace{0pt}问题分析]{{\heiti\zihao{3}\hspace{0pt}问题分析}}\label{chapter1: 问题分析}
\removelofgap
\removelotgap

\section[\hspace{-2pt}问题一]{{\heiti\zihao{-3} \hspace{-8pt}问题一}}\label{section1: 问题一}

针对问题一,我们首先需要将景点吸引力通过合适的数学函数进行量化,该函数需要综合景点可达区域、区域偏好、可容纳人数等影响。以此作为优化目标函数的一个因子。此外我们还需要考虑到不同景点、区域选择会带来不同的时间、交通成本,应当为其增加约束条件并作为优化函数的另一个因子。由于不同旅游天数会对问题产生较大的影响。一日游在晚上不必前往旅馆,而二日游、三日游都需要前往旅馆。因此我们需要额外考虑不同天数下的交通问题。

对于表3中空白的区域,我们假设可以通过中转的方式到达。比如,从景点TODO到区域TODO,以其为中转,再到景区TODO。我们应当采用Floyd‑Warshall算法计算不可直达的两景区之间的最短距离。在此基础上,采用混合整数线性优化方法得到最优方案。

我们提出一种结合景点吸引力、游客偏好、时间与空间约束的旅游路线规划模型。

\section[\hspace{-2pt}问题二]{{\heiti\zihao{-3} \hspace{-8pt}问题二}}\label{section1: 问题二}

在第二问中,当旅游套餐总数被限定在 10 种以内时,原先追求满意度极大化的模型必须在游客体验与管理简化之间做出权衡。为此,我们首先将模型提升为混合整数规划,在决策层面新增 0‑1 变量 $y_p$,通过约束 $\sum y_p\le 10$ 与“大 $M$” 逻辑联结保证仅有至多 10 条路线被激活;这一严格控制可直接给出受限情形下的全局最优,但求解规模膨胀、计算代价高。为兼顾可行性,我们在算法层面先对可行路线做预筛:依据单位满意度、覆盖度与相似度剔除明显次优或冗余方案,仅保留贡献高且互补性强的候选集进入优化;随后采用两阶段策略,先在精简集合上求线性松弛或 MILP 近优解,再按贡献度排序保留前 10 条并用贪心回填剩余游客。最终形成的综合流程是:生成全部可行路线并时长过滤,预筛精简候选,加入套餐数上限求解 MILP,如遇大规模求解超时则回退启发式结果。该折中方案既在合理计算时间内满足“套餐≤10”这一运营约束,又保证总体满意度与资源利用接近原始最优水平,实现了模型对管理需求的灵活适应。


\section[\hspace{-2pt}问题三]{{\heiti\zihao{-3} \hspace{-8pt}问题三}}\label{section1: 问题三}



