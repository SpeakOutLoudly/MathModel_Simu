\chapter[\hspace{0pt}问题分析]{{\heiti\zihao{3}\hspace{0pt}问题分析}}\label{chapter1: 问题分析}
\removelofgap
\removelotgap

\section[\hspace{-2pt}问题一]{{\heiti\zihao{-3} \hspace{-8pt}问题一}}\label{section1: 问题一}

针对问题一,我们需要根据乡村的耕地类型及作物特性,设计一个优化的农作物种植方案。首先,考虑到不同地块的种植要求,例如平旱地、梯田和山坡地只能种植一季粮食类作物,而水浇地可以种水稻或两季蔬菜。每种作物的种植面积必须满足一定的最小面积要求,并且同一地块不能连续重茬种植相同作物,这些都需要在模型中加入约束条件。

此外,根据题目要求和查阅的论文可知,乡村的露天地块每年必须种植至少401(1201/3,此处向上取整)亩豆类作物,大棚需要种植至少4亩豆类作物,这些限制也应纳入模型中进行优化。作物的销售量与价格存在不确定性,若超出销售量则会面临滞销或降价出售的风险。因此,我们在模型中需要考虑两种情景:一是滞销,二是超出部分按降价销售。

最终,优化目标是最大化乡村的农作物总收益,综合考虑作物的预期销售量、单价、产量和种植成本。在此过程中,还要考虑耕地面积的限制,确保每块地的种植面积不超过其可用面积。通过这些约束和目标,我们将能够得出一个最优的种植方案\cite{JNYZ201508097},确保乡村的农业生产既能满足市场需求,又能提高经济效。

\section[\hspace{-2pt}问题二]{{\heiti\zihao{-3} \hspace{-8pt}问题二}}\label{section1: 问题二}

在问题二中,根据经验,小麦和玉米的未来销售量预计将有$5\%$到$10\%$的年增长率,而其他作物的销售量相对于2023年则会$有±5\%$的波动。为了反映这些不确定性,我们将对每种作物的预期销售量进行随机模拟,考虑每个作物的年增长率,并将这种变化应用于模型中。这样能够准确捕捉到销售量的波动性,并对市场需求的不确定性做出调整。

农作物的亩产量也往往受到气候等因素的影响\cite{XNMI202517007},预计每年会有$±10\%$的变化。这个不确定性在模型中也需要考虑,确保我们在设定作物种植方案时能够考虑到气候变化可能带来的影响。作物的种植成本随着时间推移而逐年增长,预计年增长幅度约为$5\%$。因此,优化模型中将包括种植成本的动态增长机制,并在决策过程中考虑这一成本的增加。对于不同类型的作物,特别是粮食类作物、蔬菜类作物和食用菌类作物,销售价格也存在不同程度的波动。粮食类作物的销售价格相对稳定,而蔬菜类作物销售价格每年有大约$5\%$的增长,食用菌价格则会逐年下降,羊肚菌的销售价格每年下降幅度达到$5\%$。

基于这些不确定性和约束条件\cite{YOKE201304018},我们将利用蒙特卡洛-SAA方法对不同场景下的种植方案进行求解,最终得出2024至2030年期间的最优种植方案。这一方案不仅能够确保乡村农业生产满足市场需求,还能最大化经济效益,为乡村发展提供坚实的支持。
\section[\hspace{-2pt}问题三]{{\heiti\zihao{-3} \hspace{-8pt}问题三}}\label{section1: 问题三}


第三问的核心在于考虑农作物间的替代性和互补性\cite{1024823406.nh},并将其与价格、销售量和种植成本的相关性结合起来。这不仅要求我们在制定种植策略时考虑单一作物的利润最大化,还需要综合考虑作物之间的市场联动效应。具体而言,相同类型的作物可能在市场上存在替代关系,即一类作物的销售量增加可能会导致另一类作物的需求下降。与此同时,作物间也可能存在互补性,例如轮作和间作会提升土壤质量或作物的单产,从而促进整个种植体系的收益增长。

在这类复杂的环境下,传统的单一作物优化模型已经不能满足需求,因此需要引入基于价格和需求反向调节的模型。通过引入价格弹性系数和需求弹性系数,可以模拟作物市场中的供需变化,调整作物的价格和需求,从而更真实地反映实际市场的波动。同时,作物间的互补效应,如轮作的产量提升,也需要在模型中体现,以便根据土壤改良、作物间的协同效应优化种植决策。

此外,尽管引入了更为复杂的替代性和互补性关系,但问题的求解仍需保持在线性化范围内。通过使用适当的线性化技术(如大M法),我们可以将这些非线性关系转化为线性形式,从而使用混合整数线性规划(MILP)方法来求解。最终目标是通过模拟不同的市场情景和作物种植策略,找到2024至2030年间最优的种植方案,不仅要在收益上达到最大化,还要确保作物种植的可持续性和经济性。


