\chapter[\hspace{0pt}符号说明]{{\heiti\zihao{3}\hspace{0pt}符号说明}}\label{chapter4: 符号说明}
%--- 列格式:符号列 2 cm 居中;说明列自动伸缩;单位列 1 cm 居中 ---
\newcolumntype{C}{>{\centering\arraybackslash}m{2cm}}
\newcolumntype{U}{>{\centering\arraybackslash}m{1cm}}
\newcolumntype{D}{>{\raggedright\arraybackslash}X}



%========== 符号说明表 ==========
\begin{table}[htbp]
  \captionsetup{labelformat=empty}     % 不要自动编号
  % \caption{\heiti 四、符号说明}
  \renewcommand{\arraystretch}{1.25}   % 行距
  \small                               % 5 号字

  \begin{tabularx}{\textwidth}{|C|D|U|}
    \hline
    % \textbf{符号} & \textbf{说明} & \textbf{单位}\\
      \textbf{符号} & \textbf{说明}\\
    \hline
    %============= 表内容 =============
    $S_i$ & 第 $i$ 个景点,$i = 1,2,\dots,6$ \\
    $R_j$ & 第 $j$ 个餐饮 / 住宿区域,$j = 1,2,\dots,6$ \\
    $C_{S_i}$ & 景点 $S_i$ 容量(万人 / 半天) \\
    $L_{R_j}$ & 区域 $R_j$ 午餐接待容量(万人次 / 日中) \\
    $H_{R_j}$ & 区域 $R_j$ 晚餐 + 住宿容量(万人次 / 夜) \\
    $P_{R_j}$ & 区域 $R_j$ 游客喜好度评分 \\
    $d(S_i,R_j)$ & 景点 $S_i$ 至区域 $R_j$ 交通费用(元) \\
    $t(S_i,R_j)$ & 景点 $S_i$ 至区域 $R_j$ 交通时间(分钟) \\
    $T_{\max}$   & 半天可用于交通的最大时间(三种 $T_{\max}$) \\
    $\mathcal P_{1,2,3}$ & 一 / 二 / 三日游套餐候选集合 \\
    $\mathcal P$ & 全部可行套餐集合,$\mathcal P=\mathcal P_1\cup\mathcal P_2\cup\mathcal P_3$ \\
    $x_p$ & 选择套餐 $p$ 的游客人数(万人) \\
    $y_p$ & 0‑1 决策变量,$y_p = 1$ 表示采用套餐 $p$ \\
    $w_{\text{cost}}, w_{\text{time}}$ & 费用与时间权重系数 \\
    $\operatorname{Cost}(p)$ & 套餐 $p$ 总交通费用 \\
    $\operatorname{Time}(p)$ & 套餐 $p$ 总交通时间 \\
    $Q_i$ & 景点 $i$ 基础吸引力评分 \\
    $U_p$ & 套餐 $p$ 单位游客满意度 \\
    $A_i$ & 景点 $i$ 吸引力 \\
    $\tilde C_i$ & 容量归一化,$C_i/\max_j C_j$ \\
    $\tilde R_i$ & 服务保障度,$\min\!\bigl(1,\frac{\min(L_{R_i},H_{R_i})}{C_i}\bigr)$ \\
    $\tilde S_i$ & 喜好度归一化,$S_i/10$ \\
    $\delta_{p,i,t}$ & 若套餐 $p$ 在时段 $t$ 安排景点 $S_i$ 则为 1,否则 0 \\
    $\theta_{p,j,d}$ & 套餐 $p$ 第 $d$ 天中午选择区域 $R_j$ 用餐的 0‑1 变量 \\
    $\phi_{p,j,d}$ & 套餐 $p$ 第 $d$ 天夜晚选择区域 $R_j$ 住宿的 0‑1 变量 \\
    $C(\Delta K)$ & 扩容成本 $c_1(\Delta K)^{\gamma},\;1<\gamma\le1.2$ \\
    $\Delta K$ & 新增接待能力(万人次) \\
    $c_1$ & 扩容成本参数,$c_1 \approx 0.0007$ 亿元 \\
    $\gamma$ & 扩容指数,$\gamma \approx 1.09$ \\
    $F(\Delta K)$ & 净收益 $B(\Delta K)-C(\Delta K)$ \\
    $B(\Delta K)$ & 收益 $v_s[Z(K)-Z(K_0)] + v_p[N(K)-N(K_0)]$ \\
    $Z(K), N(K)$ & 容量为 $K$ 时的最优满意度 / 游客量 \\
    $v_s, v_p$ & 单位满意度价值 / 单位游客利润 \\
    $K^0,\;K$ & 原容量 / 扩容后容量 \\
    % ……需要更多条目就继续写……                                                                          & /   \\
    \hline
  \end{tabularx}
\end{table}


