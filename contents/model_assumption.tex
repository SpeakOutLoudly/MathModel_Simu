\chapter[\hspace{0pt}模型假设]{{\heiti\zihao{3}\hspace{0pt}模型假设}}\label{chapter2:模型假设}

\removelofgap
\removelotgap

本文针对重庆市旅游方案规划以及区域扩建规模与选址问题建立的数学模型基于以下核心假设:

\textbf{假设1:}表3中留空即为不可直达。例如,景点六只可直达区域六,但可以通过其他区域中转到达其他景点TODO。

\textbf{假设2:}从各景点到各区域酒店或餐厅的交通时间固定,不考虑现实中例如交通拥堵、车辆故障等意外情况导致的时间延长。

\textbf{假设3:}基于现实情况,本文假设从景区到区域的去程、返程交通费用与时间相等。

\textbf{假设4:}景区容量的约束在现实中不可能取决于一家旅行社的游客,因为还有可能有其他来源的若干游客,为了建模简单,把容量按元素$\times 0.6$进行折减。

\textbf{假设5:}我们假设一日游、二日游和三日游三种旅行方案相互分离,且三者在时间上并不有重叠。此时,我们只需考虑一种方案不同路线在容量上的冲突情况。

\textbf{假设6:}我们假设游客只按照对景点的喜好选择套餐,那么各个路线的人数应当假定为均等的而进行优化。