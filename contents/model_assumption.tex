\chapter[\hspace{0pt}模型假设]{{\heiti\zihao{3}\hspace{0pt}模型假设}}\label{chapter2:模型假设}

\removelofgap
\removelotgap

本文针对农作物种植优化方案以及作物间的替代性、互补性以及价格、销售量、成本等因素之间的相关性问题,建立的数学模型基于以下核心假设:

\textbf{假设1:}各类作物的需求量受市场供需的影响,且价格弹性较大。具体而言,蔬菜类作物的价格会随着供给量变化而调整,而粮食类作物的价格则保持相对刚性。

\textbf{假设2:}作物间的替代性仅限于同类作物。例如,豆类作物与豆类作物之间具有替代性,但与其他作物(如粮食作物、蔬菜等)不具备直接的替代性。

\textbf{假设3:}作物间的互补性仅限于有明确互补效应的作物对。例如,豆类作物与某些粮食作物或蔬菜作物的种植具有互补关系,互补关系通过提高作物产量和单价来体现。

\textbf{假设4:}作物的市场价格和销量之间存在价格弹性,蔬菜类作物的需求会受到价格波动的影响,而粮食类作物的需求对价格的波动较为迟钝,因此假设粮食类作物价格保持不变。

\textbf{假设5:}作物的产量在每年内保持线性增长,且轮作和互补关系所带来的增产效果是固定的。具体而言,轮作带来的增产系数为$10\%$,并且根据种植作物的不同,作物的价格和产量也会进行调整。

\textbf{假设6:}有效需求与供给量之间存在双向联动,即供给量的变化影响价格,价格变化又会反作用于需求。每个作物的产量不应超过其有效需求,且需求量随着价格调整而变化。