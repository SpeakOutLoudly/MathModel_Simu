\cqusetup{
%	************	注意	************
%	* 1. \cqusetup{}中不能出现全空的行,如果需要全空行请在行首注释
%	* 2. 不需要的配置信息可以放心地坐视不理、留空、删除或注释(都不会有影响)
%	*
%	********************************
% ===================
%	论文的中英文题目
% ===================
  ctitle = {待补充},
  etitle = {待补充},
% ===================
% 作者部分的信息
% \secretize{}为盲审标记点,在打开盲审开关时内容会自动被替换为***输出,盲审开关默认关闭
% ===================
  cauthor = \secretize{尹国伟},	% 你的姓名,以下每项都以英文逗号结束
  eauthor = \secretize{Guowei~Yin},	% 姓名拼音,~代表不会断行的空格
  studentid = \secretize{},	% 仅本科生,学号
  csupervisor = \secretize{黄~~~晟~~~~~教授},	% 导师的姓名
  esupervisor = \secretize{{Prof.~Sheng Huang}},	% 导师的姓名拼音
  cassistsupervisor = \secretize{}, % 本科生可选,助理指导教师姓名,不用时请留空为{}
  cextrasupervisor = \secretize{}, % 本科生可选,校外指导教师姓名,不用时请留空为{}
  eassistsupervisor = \secretize{}, % 本科生可选,助理指导教师或/和校外指导教师姓名拼音,不用时请留空为{}
  cpsupervisor = \secretize{}, % 仅专硕,兼职导师姓名
  epsupervisor = \secretize{},	% 仅专硕,兼职导师姓名拼音
  cclass = \secretize{\rmfamily{2025}\heiti{年}\rmfamily{5}\heiti{月}},	% 博士生和学硕填学科门类,学硕填学科类型
  research_direction = \zihao{3}{工学},
  edgree = {},	% 专硕填Professional Degree,其他按实情填写
% % 提示:如果内容太长,可以用\zihao{}命令控制字号,作用范围:{}内
  cmajor = 工~~~~学,	% 专硕不需填,填写专业名称
  emajor = , % % 专硕不需填,填写专业英文名称
  cmajora = \zihao{3}{软件工程},
  cmajorb = \zihao{3}{计算机视觉},
  cmajorc = \secretize{},
  % cmajord = 2024年6月,
% ===================
% 底部的学院名称和日期
% ===================
  cdepartment = ,	%学院名称
  edepartment = ,	%学院英文名称
% ===================
% 封面的日期可以自动生成(注释掉时),也可以解除注释手动指定,例如:二〇一六年五月
% ===================
%	mycdate = {2023年6月},
%	myedate = {June 2023},
}% End of \cqusetup
% ===================
%
% 论文的摘要
%
% ===================
\begin{cabstract}	% 中文摘要
\vspace*{-2\baselineskip}   % ← 向上拉一行
\begin{center}
  \zihao{3}\heiti{基于多目标优化的种植方案研究}\\[0.8em]
\end{center}

\begin{center}
  \zihao{3}\heiti{摘要}\\[0.8em]
\end{center}

为推动有限耕地资源的高效利用,保障粮食安全与乡村经济的可持续发展,本文以位于华北山区的某乡村为研究对象,对2024—2030年多年度农作物种植规划问题进行了系统化研究。该乡村气候偏冷,耕地分为平旱地、梯田、山坡地和水浇地四种类型,以及普通大棚与智慧大棚两类设施用地。不同地类适宜的作物种类及种植季节各不相同,同时受限于单地块不能连续重茬同种作物、每三年必须至少种植一次豆类作物等轮作制度约束。此外,种植方案需兼顾田间管理便利性,包括种植地块的集中程度与单块面积下限等要求。市场因素方面,各类作物的产量、需求量、价格及种植成本存在波动性和相互关联性,超过需求部分的处理方式也会显著影响收益。

在问题一中,假设未来各作物的亩产量、销售量、价格和成本均保持稳定,构建以各地块、各季节作物种植面积为核心决策变量的混合整数线性规划模型,综合考虑土地类型适配、产销平衡、轮作约束及最小面积限制等条件。在此基础上,分别针对超过销售量部分完全滞销与按原价五折销售两种情形,给出2024—2030年的年度最优种植方案,结果以表格形式输出,为乡村制定长期生产计划提供直接参考。

在问题二中,放宽稳定性假设,引入未来多年内作物需求量、亩产量、成本与价格的年度变化特征,其中部分作物需求呈增长趋势,部分存在上下波动,产量受气候影响波动,成本与价格则呈不同幅度的变化趋势。通过设定波动范围与概率分布,利用蒙特卡洛方法生成多场景数据,并采用样本平均近似方法(SAA)进行优化求解,得到兼顾预期收益与风险控制的稳健种植方案。

在问题三中,进一步考虑市场的复杂互动机制,包括不同作物之间的可替代性与互补性、价格与需求之间的联动关系,以及豆类作物轮作对其他作物产量的促进作用。通过构建相关性矩阵与市场反馈函数,对需求和价格在种植结构调整下的动态变化进行模拟,并在优化过程中嵌入迭代更新机制,得到能够适应市场变化的最优种植策略。最后,将该策略与问题二中基于独立波动假设的方案进行对比分析,揭示市场互动对种植结构、收益水平与风险分布的影响规律。
\end{cabstract}
% 中文关键词,请使用英文逗号分隔:
\ckeywords{MILP;蒙特卡洛;种植方案优化;作物间替代性与互补性}

% 封面和摘要配置完成