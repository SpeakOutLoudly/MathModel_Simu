\cqusetup{
%	************	注意	************
%	* 1. \cqusetup{}中不能出现全空的行,如果需要全空行请在行首注释
%	* 2. 不需要的配置信息可以放心地坐视不理、留空、删除或注释(都不会有影响)
%	*
%	********************************
% ===================
%	论文的中英文题目
% ===================
  ctitle = {待补充},
  etitle = {待补充},
% ===================
% 作者部分的信息
% \secretize{}为盲审标记点,在打开盲审开关时内容会自动被替换为***输出,盲审开关默认关闭
% ===================
  cauthor = \secretize{尹国伟},	% 你的姓名,以下每项都以英文逗号结束
  eauthor = \secretize{Guowei~Yin},	% 姓名拼音,~代表不会断行的空格
  studentid = \secretize{},	% 仅本科生,学号
  csupervisor = \secretize{黄~~~晟~~~~~教授},	% 导师的姓名
  esupervisor = \secretize{{Prof.~Sheng Huang}},	% 导师的姓名拼音
  cassistsupervisor = \secretize{}, % 本科生可选,助理指导教师姓名,不用时请留空为{}
  cextrasupervisor = \secretize{}, % 本科生可选,校外指导教师姓名,不用时请留空为{}
  eassistsupervisor = \secretize{}, % 本科生可选,助理指导教师或/和校外指导教师姓名拼音,不用时请留空为{}
  cpsupervisor = \secretize{}, % 仅专硕,兼职导师姓名
  epsupervisor = \secretize{},	% 仅专硕,兼职导师姓名拼音
  cclass = \secretize{\rmfamily{2025}\heiti{年}\rmfamily{5}\heiti{月}},	% 博士生和学硕填学科门类,学硕填学科类型
  research_direction = \zihao{3}{工学},
  edgree = {},	% 专硕填Professional Degree,其他按实情填写
% % 提示:如果内容太长,可以用\zihao{}命令控制字号,作用范围:{}内
  cmajor = 工~~~~学,	% 专硕不需填,填写专业名称
  emajor = , % % 专硕不需填,填写专业英文名称
  cmajora = \zihao{3}{软件工程},
  cmajorb = \zihao{3}{计算机视觉},
  cmajorc = \secretize{},
  % cmajord = 2024年6月,
% ===================
% 底部的学院名称和日期
% ===================
  cdepartment = ,	%学院名称
  edepartment = ,	%学院英文名称
% ===================
% 封面的日期可以自动生成(注释掉时),也可以解除注释手动指定,例如:二〇一六年五月
% ===================
%	mycdate = {2023年6月},
%	myedate = {June 2023},
}% End of \cqusetup
% ===================
%
% 论文的摘要
%
% ===================
\begin{cabstract}	% 中文摘要
\vspace*{-2\baselineskip}   % ← 向上拉一行
\begin{center}
  \zihao{3}\heiti{基于多目标优化的旅游线路规划研究}\\[0.8em]
\end{center}

\begin{center}
  \zihao{3}\heiti{摘要}\\[0.8em]
\end{center}

% 摘要第一段可以不用开门见山直接“问题一。。。”,感觉可以根据后续内容重复率进行取舍
针对问题一,我们首先利用 Floyd-Warshall 算法补全景点—区域交通网络,生成任意节点对的最短路矩阵,由此精确计算交通时间 \(t(S_i,R_j)\) 与费用 \(d(S_i,R_j)\)。在此基础上,将景点基准吸引力、游客对餐饮/住宿区域的偏好 \(P_{R_j}\) 与交通成本共同纳入满意度模型,构造 \text{Pref}(p)、\text{Cost}(p)、\text{Time}(p) 等函数,并归一化形成综合效用 \(u_p\)。通过枚举一日、二日、三日游所有可行路线,引入 MILP 处理景点、午餐及夜宿容量约束,再结合遗传算法对解空间进行多样化搜索,获得最短时间、最低费用及均衡型等多类别最优方案。最终输出 TODO 条覆盖不同天数、不同兴趣偏好的高满意度套餐。

针对问题二,问题一中生成的旅游方案存在着数量较多、部分方案相似度较高等问题。因此第二问针对这些问题,提出基于旅游方案相似度的加权距离函数建模,并采用 K-means++ 聚类,将套餐数压缩至 10 种以内且总体方案满意度较高。加权距离函数主要由景点吸引力、区域吸引力、路线相似度、交通时间、交通费用、游客偏好等因素构成。算法会尝试不同K值,并根据簇内平方和选择最优的K值作为最终的结果。我们将每个簇中满意度最高的方案保留,最终输出满足数量上限且整体满意度最优的方案集合。

针对问题三,我们建立了扩容收益、建设成本等因素,选择最合适的区域以及扩容规模。我们首先根据题目背景,引入扩容收益、建设成本等因素,得到净收益函数。然后,我们采用双层优化策略,外层枚举六个候选区域及一系列离散的扩容步长,内层优化在给定 $(r,\Delta K)$ 时,调用问题一中“枚举路线 + MILP”框架(辅以遗传算法与启发式多样化规则)重新分配游客,得到 $Z_r(K),\,N_r(K)$。最终,我们绘制 $F_r(\Delta K)$ 随扩容规模变化的曲线,并给出最优扩容区域及规模。
\end{cabstract}
% 中文关键词,请使用英文逗号分隔:
\ckeywords{MILP;遗传算法;旅游线路规划;K均值聚类}

% 封面和摘要配置完成