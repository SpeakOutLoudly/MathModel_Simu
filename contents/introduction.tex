\chapter[\hspace{0pt}问题重述]{{\heiti\zihao{3}\hspace{0pt}问题重述}}\label{chapter1: 问题重述}
\removelofgap
\removelotgap
\setcounter{page}{2}  % 自行修改

\section[\hspace{-2pt}问题背景]{{\heiti\zihao{-3} \hspace{-8pt}问题背景}}\label{section1: 问题背景}

重庆作为著名的网红城市,前来旅游的乘客数量极大。由于重庆山河交错的地理环境、组团式城市形态的客观原因,景点多但是分布分散且单一景点资源有限。此外重庆主城区域面积有限但接收大量游客。如何通过规划旅游路线减少交通时间,防止景点人员拥挤以提升游客旅游体验是一个具有实际意义的优化问题。

根据题目提供的相关数据可知:共有6个景点并且景点可接纳人数在1.2万人到4.2万人之间;景点附近有6个主要区域并为游客提供午餐、晚餐和住宿,其接待能力和游客喜好程度可见表 2。游客会在上午、下午游览景点,中午到附近区域用餐,若时间超过一天则选择一个区域食宿。题目还给出了具体的从景点到各区域的交通费用和时间。对此,我们需要建立模型解决以下问题:

\section[\hspace{-2pt}问题重述]{{\heiti\zihao{-3} \hspace{-8pt}问题重述}}\label{section1: 问题重述}

为了为游客提供更好的方案选择,并且提高每个方案的旅游体验。我们需要综合考虑以上所有信息,并注意一些基本原则:

\textbf{问题一:}我们需要考虑景点和附近区域的接待能力,保证同一时间在任意景点或区域的总人数不会超过其接纳上线。并且要考虑到交通费用与时间,使游客能够尽量节省时间与金钱。在此要求下,我们根据不同旅游时间需要提出景区吸引力函数,结合约束条件设计出合适的“景点+区域”组合方案。

% 综合考虑景点与区域的容量限制、交通耗时与费用以及游客偏好,构建景区吸引力函数,在保证任意时段游客量不超载的基础上,为一日游、二日游与三日游分别设计若干“景点 + 区域”套餐方案,并给出最优游客分配。

\textbf{问题二:}过多的旅游方案会导致更大的路线管理压力以及更低的游客决策效率。我们需要将问题一中的套餐总数控制在10以内。由此,我们要建立方案筛选与整合准则,对原方案进行精简与再优化,将重复性较强的方案合并,最终输出满足数量上限且整体满意度最优的套餐集合。

\textbf{问题三:}我们需要考虑区域的接纳人数、游客喜好程度等信息,尽可能为旅游集团提出效益更高的建设方案。我们需要提出一种评价函数结合扩容收益、建设成本等因素,选择最合适的区域以及扩容规模。

