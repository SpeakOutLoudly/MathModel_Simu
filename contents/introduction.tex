\chapter[\hspace{0pt}问题重述]{{\heiti\zihao{3}\hspace{0pt}问题重述}}\label{chapter1: 问题重述}
\removelofgap
\removelotgap
\setcounter{page}{2}  % 自行修改

\section[\hspace{-2pt}问题背景]{{\heiti\zihao{-3} \hspace{-8pt}问题背景}}\label{section1: 问题背景}

随着中国乡村振兴战略的深入推进,农业产业结构优化和可持续发展成为乡村经济振兴的重要课题。华北山区某乡村耕地资源有限且分散,现有1201亩露天耕地和12亩大棚。受气候限制,大多数耕地每年仅能种植一季作物,且不同地块适宜种植的作物类型不同。此外,农作物种植需遵循轮作规则和豆类种植要求,同时需兼顾田间管理的便利性。

而2023年的种植数据为优化未来种植策略提供了依据。未来,农作物的预期销量、亩产量、成本和价格可能受市场、气候等因素影响而波动。例如,小麦、玉米需求可能增长,蔬菜价格可能上涨,而食用菌价格可能下降。 我们需要综合考虑耕地资源约束、农作物生长规律、市场需求变化和经济效益最大化等因素,建立模型解决以下问题:

\section[\hspace{-2pt}问题重述]{{\heiti\zihao{-3} \hspace{-8pt}问题重述}}\label{section1: 问题重述}

为了优化农作物种植结构,提高土地利用效率并实现经济效益最大化,我们需要综合考虑耕地资源、作物生长规律和市场因素,并遵循以下基本原则:

\textbf{问题一:}在假设未来农作物预期销售量、种植成本、亩产量和销售价格保持稳定的前提下,我们需要制定2024-2030年的最优种植方案。当产量超过预期销售量时,该问题需考虑超产部分完全滞销造成浪费,超产部分可按2023年销售价格的$50\%$降价出售两种不同的销售情景。

\textbf{问题二:}考虑到实际农业生产中存在诸多不确定性因素,我们需要建立鲁棒性优化模型,在考虑考虑市场需求、产量波动、成本上涨及价格变动等不确定性因素的情况下,重新制定出2024-2030年的最优种植方案。

\textbf{问题三:}在问题二的基础上,需要进一步考虑农作物之间的可替代性和互补性,以及销售量、价格和成本之间的相关性。我们需要建立更符合现实情况的综合评价模型,通过模拟分析得到最优种植策略,并与问题二的结果进行对比分析,为实际决策提供更全面的参考依据。
