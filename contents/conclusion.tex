\chapter[\hspace{0pt}模型评价与推广]{{\heiti\zihao{3}\hspace{0pt}模型评价与推广}}\label{chapter4: 模型评价与推广}
\removelofgap
\removelotgap

\section[\hspace{-2pt}主要结论]{{\heiti\zihao{-3} \hspace{-8pt}主要结论}}\label{section5: 主要结论}

\subsection[\hspace{-2pt}问题一:旅游路线智能分配优化]{{\heiti\zihao{4} \hspace{-8pt}问题一:旅游路线智能分配优化}}\label{subsection5: 问题一评价}

通过构建基于混合整数线性规划的多目标优化模型,成功实现了重庆旅游路线的智能分配,主要取得以下成果:

\noindent\textbf{多目标优化效果显著}:基于费用、时间和满意度的多目标优化模型能够有效平衡各项指标。实验结果表明,不同权重配置下的优化方案都能适应不同游客需求,其中费用优先型配置在三日游方案中表现最优,总费用降低至341.32万元,平均满意度提升至0.36。

\noindent\textbf{大规模游客分流能力强}:模型能够有效处理大规模游客分流问题,在一日游方案中成功分配3万游客到2条最优路线,二日游方案分配3万游客到4条路线,三日游方案分配2万游客到4条路线。通过容量约束和时段复用机制,实现了景点和区域资源的充分利用。

\noindent\textbf{算法融合策略有效}:采用Floyd-Warshall算法预处理交通矩阵,结合遗传算法和启发式方法生成候选路线,最终通过混合整数线性规划求解最优分配。混合方法成功生成1200+条高质量候选路线,证明了多算法融合策略在大规模优化问题中的有效性。

\section[\hspace{-2pt}模型优点]{{\heiti\zihao{-3} \hspace{-8pt}模型优点}}\label{section5: 模型优点}

\subsection[\hspace{-2pt}问题一模型优点]{{\heiti\zihao{4} \hspace{-8pt}问题一模型优点}}\label{subsection5: 问题一模型优点}

\noindent\textbf{理论基础扎实}:模型基于混合整数线性规划(MILP)理论构建,具有严格的数学理论支撑。通过线性目标函数和约束条件的设计,保证了求解的最优性和唯一性。同时,采用加权综合评价方法,将费用、时间和满意度等多维指标统一到同一评价体系中。

\noindent\textbf{技术方法先进}:集成了多种先进的优化算法,包括Floyd-Warshall最短路径算法用于交通网络预处理,遗传算法(GA)用于高质量候选解生成,启发式算法用于解空间探索,以及MILP用于精确求解。这种多算法融合的策略有效提升了模型的求解效率和解质量。

\noindent\textbf{实用价值突出}:模型能够处理包含6个景点、6个区域的实际规模旅游网络,支持1-3日游多种旅游方案,具备强大的游客分流能力。实验验证表明,模型可同时优化分配2-4万游客的路线安排,为旅游管理部门提供了实用的决策支持工具。

\section[\hspace{-2pt}不足与改进方向]{{\heiti\zihao{-3} \hspace{-8pt}不足与改进方向}}\label{section5: 不足与改进方向}

\subsection[\hspace{-2pt}问题一模型的主要局限]{{\heiti\zihao{4} \hspace{-8pt}问题一模型的主要局限}}\label{subsection5: 问题一模型局限}

\noindent\textbf{容量约束简化}:当前模型采用时段复用的简化处理方式,假设景点和区域容量可在不同时段完全复用,但实际情况中可能存在时段冲突和资源竞争问题。

\noindent\textbf{路线生成限制}:为控制计算复杂度,对二日游和三日游候选路线数量进行了限制(分别为5000和1200+条),可能遗漏部分高质量解。

\noindent\textbf{满意度建模简化}:满意度指标主要基于区域偏好度的线性组合,未充分考虑游客个性化需求、天气、季节等动态因素的影响。

\subsection[\hspace{-2pt}问题一模型的改进方向]{{\heiti\zihao{4} \hspace{-8pt}问题一模型的改进方向}}\label{subsection5: 问题一模型改进}

\noindent\textbf{动态优化机制}:引入实时数据更新和动态调整机制,根据实际游客流量、天气变化等因素动态调整路线分配,提高模型的适应性和实用性。

\noindent\textbf{精细化约束建模}:建立更精细的时空约束模型,考虑景点的实际开放时间、交通拥堵状况、季节性容量变化等因素,提高约束建模的准确性。

\noindent\textbf{智能化程度提升}:结合机器学习方法,如深度强化学习、协同过滤等技术,构建更智能的游客偏好预测模型和个性化推荐系统,进一步提升旅游体验质量。

\subsection[\hspace{-2pt}问题二模型的局限与改进方向]{{\heiti\zihao{4}\hspace{-8pt}问题二模型的局限与改进方向}}\label{subsection6: 问题二模型改进}

\noindent\textbf{簇数判定单一}:当前 K\,means 仅以惯性(SSE)为准则确定 $K$,未结合轮廓系数、Gap Statistic 等综合指标,存在簇数局部最优风险。\\[2pt]
\noindent\textbf{特征权重经验化}:距离函数中六类权重 $w_k$ 依赖经验设定,缺乏数据驱动的自动标定,可能导致聚类边界受主观偏差影响。\\[2pt]
\noindent\textbf{簇内多样性丢失}:每簇仅保留“净收益最高”路线,忽略簇内其他差异化方案,易降低最终套餐的多样性与弹性。\\[2pt]
\noindent\textbf{对容量耦合弱化}:聚类阶段未显式考虑景点与区域剩余容量,可能在后续分配时再次出现容量瓶颈需人工回调。\\[6pt]

\noindent\textbf{改进方向}\\
\noindent\textbf{(1)多指标确定 $K$}:引入轮廓系数、Calinski-Harabasz 分数与 Gap Statistic 共同评估,采取“肘部+稳健性”策略自动选择最佳簇数。\\
\noindent\textbf{(2)权重学习机制}:利用历史游客反馈或专家打分,通过多目标贝叶斯优化或熵权法自适应确定 $w_k$,提升距离度量客观性。\\
\noindent\textbf{(3)簇内代表线路扩充}:对每簇设置多样性阈值,保留 1–2 条距离中心较近但风格互补的次优路线,以丰富套餐选择。\\

\subsection[\hspace{-2pt}问题三模型的局限与改进方向]{{\heiti\zihao{4}\hspace{-8pt}问题三模型的局限与改进方向}}\label{subsection7: 问题三模型改进}

\noindent\textbf{成本参数不确定}:扩容成本函数 $C(\Delta K)=c_1(\Delta K)^{\gamma}$ 基于宏观均值估计,未区分土地级差、层高限制等实际建造条件。\\[2pt]
\noindent\textbf{枚举步长粗粒度}:外层离散搜索步长以 0.5 万—1 万人次为间隔,可能错过更精细的最佳规模。\\[2pt]
\noindent\textbf{单期静态优化}:仅评估一次性建成带来的净现值,未将投资分期、建设周期与资金时间价值纳入 NPV 分析。\\[6pt]

\noindent\textbf{改进方向}\\
\noindent\textbf{(1)多情景成本估计}:对 $c_1,\gamma$ 设置信度区间,采用蒙特卡洛模拟评估成本不确定性对最优规模的影响,得到稳健决策区间。\\
\noindent\textbf{(2)引入需求-价格弹性}:将扩容后门票/酒店价格调整与新增客流弹性纳入收益函数 $B(\Delta K)$,更贴近市场机制。\\
\noindent\textbf{(3)自适应网格搜索}:先粗粒度扫描,再在收益峰值附近采用局部细粒度迭代(如黄金分割搜索),提高规模精度并压缩计算量。\\
