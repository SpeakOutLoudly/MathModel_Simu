\chapter[\hspace{0pt}模型评价与推广]{{\heiti\zihao{3}\hspace{0pt}模型评价与推广}}\label{chapter4: 模型评价与推广}
\removelofgap
\removelotgap

\section[\hspace{-2pt}主要结论]{{\heiti\zihao{-3} \hspace{-8pt}主要结论}}\label{section5: 主要结论}

通过构建基于混合整数线性规划的多目标优化模型,成功实现了重庆旅游路线的智能分配,主要取得以下成果:

\noindent\textbf{多目标优化效果显著}:基于费用、时间和满意度的多目标优化模型能够有效平衡各项指标。实验结果表明,不同权重配置下的优化方案都能适应不同游客需求,其中费用优先型配置在三日游方案中表现最优,总费用降低至341.32万元,平均满意度提升至0.36。

\noindent\textbf{算法融合策略有效}:采用Floyd-Warshall算法预处理交通矩阵,结合遗传算法和启发式方法生成候选路线,最终通过混合整数线性规划求解最优分配。混合方法成功生成1200+条高质量候选路线,证明了多算法融合策略在大规模优化问题中的有效性。

\section[\hspace{-2pt}模型优点]{{\heiti\zihao{-3} \hspace{-8pt}模型优点}}\label{section5: 模型优点}

\subsection[\hspace{-2pt}问题一模型优点]{{\heiti\zihao{4} \hspace{-8pt}问题一模型优点}}\label{subsection5: 问题一模型优点}

\noindent\textbf{理论基础扎实}:模型基于混合整数线性规划(MILP)理论构建,具有严格的数学理论支撑。通过线性目标函数和约束条件的设计,保证了求解的最优性和唯一性。同时,采用加权综合评价方法,将费用、时间和满意度等多维指标统一到同一评价体系中。

\noindent\textbf{技术方法先进}:集成了多种先进的优化算法,包括Floyd-Warshall最短路径算法用于交通网络预处理,遗传算法(GA)用于高质量候选解生成,启发式算法用于解空间探索,以及MILP用于精确求解。这种多算法融合的策略有效提升了模型的求解效率和解质量。

\subsection[\hspace{-2pt}问题二模型优点]{{\heiti\zihao{4} \hspace{-8pt}问题二模型优点}}\label{subsection5: 问题二模型优点}

\noindent\textbf{简洁高效}:自定义距离函数+K\,means 仅需秒级计算,即可将百级候选路线压缩为 $\le 10$ 条代表方案。

\noindent\textbf{多维综合}:同时考虑景点、区域、费用、时间与偏好等 6 类特征,保证聚类结果在多维度上“互补且离散”。

\noindent\textbf{可迁移性强}:更新特征即可复用于其他城市或新景点,无需更改算法主体。

\subsection[\hspace{-2pt}问题三模型优点]{{\heiti\zihao{4} \hspace{-8pt}问题三模型优点}}\label{subsection5: 问题三模型优点}

\noindent\textbf{量化决策}:模型能够将复杂的扩容决策量化为净收益,为旅游集团提供了清晰的经济指标,使得决策过程更加客观和科学。

\noindent\textbf{多区域比较}:模型同时考虑了多个区域的扩容潜力,并给出了各区域在不同扩容规模下的表现,有助于全面评估不同选项。

\noindent\textbf{考虑边际效益}:结果显示净收益并非随扩容规模线性增长,而是存在最优解,模型考虑了扩容的边际成本和边际收益,这更符合实际情况。
\section[\hspace{-2pt}不足与改进方向]{{\heiti\zihao{-3} \hspace{-8pt}不足与改进方向}}\label{section5: 不足与改进方向}

\subsection[\hspace{-2pt}问题一模型的主要局限]{{\heiti\zihao{4} \hspace{-8pt}问题一模型的主要局限}}\label{subsection5: 问题一模型局限}

\noindent\textbf{容量约束简化}:当前模型采用时段复用的简化处理方式,假设景点和区域容量可在不同时段完全复用,但实际情况中可能存在时段冲突和资源竞争问题。

\noindent\textbf{路线生成限制}:为控制计算复杂度,对二日游和三日游候选路线数量进行了限制(分别为5000和1200+条),可能遗漏部分高质量解。


\subsection[\hspace{-2pt}问题一模型的改进方向]{{\heiti\zihao{4} \hspace{-8pt}问题一模型的改进方向}}\label{subsection5: 问题一模型改进}

\noindent\textbf{动态优化机制}:引入实时数据更新和动态调整机制,根据实际游客流量、天气变化等因素动态调整路线分配,提高模型的适应性和实用性。

\noindent\textbf{精细化约束建模}:建立更精细的时空约束模型,考虑景点的实际开放时间、交通拥堵状况、季节性容量变化等因素,提高约束建模的准确性。


\subsection[\hspace{-2pt}问题二模型的局限与改进方向]{{\heiti\zihao{4}\hspace{-8pt}问题二模型的局限与改进方向}}\label{subsection6: 问题二模型改进}

\noindent\textbf{簇数判定单一}:当前 K\,means 仅以惯性(SSE)为准则确定 $K$,未结合轮廓系数、Gap Statistic 等综合指标,存在簇数局部最优风险。\\[2pt]
\noindent\textbf{特征权重经验化}:距离函数中六类权重 $w_k$ 依赖经验设定,缺乏数据驱动的自动标定,可能导致聚类边界受主观偏差影响。\\[2pt]

\noindent\textbf{改进方向}\\
\noindent\textbf{(1)多指标确定 $K$}:引入轮廓系数、Calinski-Harabasz 分数与 Gap Statistic 共同评估,采取“肘部+稳健性”策略自动选择最佳簇数。\\
\noindent\textbf{(2)权重学习机制}:利用历史游客反馈或专家打分,通过多目标贝叶斯优化或熵权法自适应确定 $w_k$,提升距离度量客观性。\\


\subsection[\hspace{-2pt}问题三模型的主要局限]{{\heiti\zihao{4} \hspace{-8pt}问题三模型的主要局限}}\label{subsection5: 问题三模型局限}


\noindent\textbf{未考虑投资回报期}:模型给出的净收益是特定扩容规模下的一个结果,但没有提及投资回收期或长期运营的财务状况。高净收益的方案可能需要巨大的初始投资,而投资回收期过长可能会影响决策。

\noindent\textbf{成本参数不确定}:扩容成本函数 $C(\Delta K)=c_1(\Delta K)^{\gamma}$ 基于宏观均值估计,未区分土地级差、层高限制等实际建造条件。\\[2pt]

\subsection[\hspace{-2pt}问题三模型的改进方向]{{\heiti\zihao{4} \hspace{-8pt}问题三模型的改进方向}}\label{subsection5: 问题三模型改进}


\noindent\textbf{投资回报评估}:引入时间维度,计算投资回报期、净现值(NPV)或内部收益率(IRR)等财务指标,以评估项目的长期盈利能力和风险。

\noindent\textbf{多情景成本估计}:对 $c_1,\gamma$ 设置信度区间,采用蒙特卡洛模拟评估成本不确定性对最优规模的影响,得到稳健决策区间。\\



